% The very first thing in every latex file is the definition of the document class, in this case scrbook. There is also e.g. article, report, letter, beamer... The arguments in the square-brackets are optional arguments defining e.g. the font size (11pt) or the margins (DIV, BCOR).
\documentclass[11pt,bibliography=totoc,openany,numbers=noendperiod]{scrbook}

% Import of packages and options regarding the whole document are outsourced in preamble.sty, which is a so-called style-file (that's why the -.sty ending). So similar to e.g. "import numpy" in Python, there is many packages which are loaded in there. So let's have a look at it...
\usepackage{preamble}
\bibliographystyle{apalike}	% the style how the references are written in the list of references, see https://www.overleaf.com/learn/latex/Natbib_bibliography_styles 
   
% Begin of the main body of your document. This command must be followed by an \end{document} command at the very end of the document.
\begin{document}

% title page
\begin{titlepage}		% titlepage is a special environment, which is defined between this command and the corresponding \end{titlepage} command

	% suppress error "destination with the same identifier" ...
  \setcounter{page}{-1}			% to have the page counter showing 1 for the first actual page after the titlepages

\begin{center}


% Upper part of the page

\textsc{\Large Polar Ocean Climate AGF-214}\\[0.5cm]


% Title
\newcommand{\HRule}{\rule{\linewidth}{0.5mm}}
\HRule \\[0.4cm]
{ \huge \bfseries Field Report 2022}\\
\HRule \\[1.5cm]
\includegraphics[width=0.8\textwidth]{./figures_main/cover_image} % if there are more than just one image, they should be merged in a graphic program like GIMP beforehand.

\vfill

% Author and supervisor

\emph{Authors:}\\
John Lennon, Paul McCartney, Mary Poppins, Mark Twain

\vspace{1cm}

\emph{Instructors:}\\
Moby Dick, Nils Holgerson


\vfill

% Bottom of the page
\includegraphics[width=0.3\textwidth]{./figures_main/unis_logo}

\end{center}

\newpage
\thispagestyle{empty}			% to suppress the header and footer lines for the title pages

\end{titlepage}



% tables (table of contents, abbreviations, list of figures, abstract, ...)
% start preamble --> page numbering roman
\frontmatter

\addcontentsline{toc}{chapter}{Preface}

\chapter*{Preface}

\lipsum

% table of contents
\tableofcontents		% this command automatically creates a toc from the section names you specify further down, see https://www.overleaf.com/learn/latex/Table_of_contents
\cleardoublepage		% empty pages to separate the table of contents	from the rest


% Introductory part: might not be needed for all courses, delete if not applicable
\backmatter % even though it is not in the back, the backmatter formatting can be used to distinguish this part from the main scientific chapters
\chapter{Introduction}

\section{The Nordic Seas}

\textit{John Lennon \& Paul McCartney}

\vspace{1cm}

\lipsum[1] \citep{stull1988introduction}

\vspace{3cm}
\bibliography{biblio_john}
\bibliographystyle{apalike}

% main part --> page numbering arabic
\mainmatter


\chapter{Turbulence in the Arctic}
\label{john:turbulence}

\begin{center}								% center the text horizontally

\textit{John Lennon \& Paul McCartney}				% names of all authors

\vspace{1cm}

\begin{minipage}{0.9\textwidth}				% minipage with a width of 0.9 times the total text width, just a matter of personal taste
\centerline{\textbf{\large Abstract}}
\vspace{0.3cm}
\lipsum[1]									% abstract text

\end{minipage}
\end{center}

\vspace{1cm}

\section{Introduction}		% the next smaller level after chapter is section...
\label{john:introduction}
\lipsum[1] \citep{nilsen2016simple}.		% latex style to cite a paper from your .bib file. two ways: \citet{•} puts the brackets only around the year, e.g. "I've read it in Einstein et al., (1915)." \citep{•} puts the brackets around everything, like you did it most of the times..., see https://www.overleaf.com/learn/latex/natbib_citation_styles

\begin{figure}[htbp]		% typical way to include a figure. The option [htbp] means, that the compiler should try to place it Here first (that's why h first), then at the Top of the next page (t as second letter), at the bottom of the page (b) and as last option wherever it fits best...
    \centering				% center the figure horizontally
    \includegraphics[width=\textwidth]{./figures_john/schematic_ABL.png}
    \vspace{-20pt}		% put the caption a little closer to the figure
    \caption[Short description for list of figures]{Longer description to be printed under the figure, can include references like \citep{stull1988introduction}} % The text in the curly brackets is what appears below the figure, the text in the square brackets appears in the list of figures. In that way, you can give it a shorter description for the list and a longer text to put actually with the figure.
	\label{john:fig1}		% label to reference to the figure in the text (I've learned, that you need to refer to every figure in your document at least once... You should doublecheck that with your supervisor)
\end{figure}

\lipsum



\section{Theory}
\label{john:theory}
\subsection{Turbulent fluxes}
\label{john:fluxes}
\lipsum


\section{Data \& Methods}
\label{john:methods}
\lipsum

\section{Results}
\label{john:results}
\lipsum

% example for a table:
\begin{table}[htbp]
    \setlength\belowcaptionskip{1.5ex}
    \centering
    \caption[Stability regime definitions]{Definitions and total occurrence frequencies (during Period One) of the stability regime used within this study.}
    \label{john:table_stab_regimes}
    \begin{tabular}{|c|c|c|c|}
    \hline
    \textbf{Regime} & \textbf{Abbreviation} & \textbf{range of MOST} & \textbf{occurrence frequency/} \\
     & & \textbf{stability parameter} & \textbf{total data points} \\ \hline
    unstable & us & $\zeta < 0.0$ & 27.09\,\% / 11\,808 \\ \hline
    nearly neutral & nn & $0.0 \leq \zeta < 0.02$ & 15.78\,\% / 6\,878 \\ \hline
    weakly stable & ws & $0.02 \leq \zeta < 0.7$ & 52.25\,\% / 22\,776 \\ \hline
    very stable & vs & $0.7 \leq \zeta < 3.0$ & 4.20\,\% / 1\,833 \\ \hline
    extremely stable & es & $\zeta \geq 3.0$ & 0.69\,\% / 299 \\ \hline
    \end{tabular}
\end{table}

Reference to Table \ref{john:table_stab_regimes} looks like this ...


\section{Discussion}
\label{john:discussion}
\lipsum

\section{Conclusions \& Outlook}
\label{john:conclusion}
\lipsum

\vspace{3cm}
% lists for figures, tables, literature...
\bibliography{biblio_john}
\bibliographystyle{apalike}				% include the the different chapters, which are saved in separate files
% ...


\cleardoublepage



% appendix
\appendix								% change style to Appendix
\chapter{Appendix}

\section{CTD}

\textit{John Lennon \& Paul McCartney}

\vspace{1cm}

\lipsum[1] \citep{stull1988introduction}

\vspace{3cm}
\bibliography{biblio_john}
\bibliographystyle{apalike}

    
\end{document}